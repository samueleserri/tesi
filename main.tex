\documentclass[12pt]{book}

% Page layout
\usepackage[a4paper,margin=3cm]{geometry}

% Encoding and language
\usepackage[utf8]{inputenc}
\usepackage[T1]{fontenc}
\usepackage{newunicodechar}
\newunicodechar{ℓ}{\ensuremath{\ell}}
\usepackage[english]{babel}
\usepackage{lmodern}
\usepackage{microtype}

% Math and graphics
\usepackage{amsmath,amssymb}
\usepackage{graphicx}
\usepackage{tikz}
\usepackage{float}
% code
\usepackage{listings}
\lstset{language=Python,
        basicstyle=\ttfamily\small,
        keywordstyle=\color{blue},
        commentstyle=\color{green!60!black},
        stringstyle=\color{orange},
        showstringspaces=false,
        numbers=left,
        numberstyle=\tiny,
        stepnumber=1,
        numbersep=5pt,
        breaklines=true,
        frame=single,
        tabsize=4,
        captionpos=b,
        morekeywords={self, None, True, False}
}
% pseudocode
\usepackage{algorithm}
\usepackage{algpseudocode}
% Links
\usepackage{csquotes}
\usepackage{hyperref}
\hypersetup{
    hidelinks,
    colorlinks=true,
    linkcolor=black,
    urlcolor=black,
    citecolor=green,
    filecolor=black,
    breaklinks=true  
}

% add this to make page numbers always bottom center
\usepackage{fancyhdr}
\setlength{\headheight}{14.49998pt}  % fix fancyhdr warning
\pagestyle{fancy}
\fancyhf{}                      % clear all header/footer fields
\fancyhead[L]{\leftmark}        % chapter name in left header
\fancyhead[R]{\thepage}         % page number in right header
\renewcommand{\headrulewidth}{0.4pt}  % thin line under header
\renewcommand{\footrulewidth}{0pt}

% customize the plain page style (used on chapter-start pages)
\fancypagestyle{plain}{%
  \fancyhf{}                    % clear all header/footer fields
  \fancyfoot[C]{\thepage}       % page number centered in footer only
  \renewcommand{\headrulewidth}{0pt}  % no header line
  \renewcommand{\footrulewidth}{0pt}
}

\usepackage[backend=bibtex,style=numeric-comp,sorting=nyt]{biblatex} % use bibtex backend to avoid biber
\addbibresource{references.bib} % The file containing the bibliography
% Simple metadata
\newcommand{\thesistitle}{TITOLO}
\newcommand{\thesisauthor}{Samuele Serri}
\newcommand{\thesismatricola}{SM3201369}
\newcommand{\thesissupervisor}{Prof. Nome Cognome}
\newcommand{\thesisyear}{2024--2025}

% useful math commands
\newcommand{\R}{\mathbb{R}}
\newcommand{\N}{\mathbb{N}}

% create math environment for theorems, definitions, etc.
\usepackage{amsthm}
\newtheorem{definition}{Definition}[chapter]
\newtheorem{theorem}{Theorem}[chapter]
\newtheorem{lemma}{Lemma}[chapter]
\newtheorem{proposition}{Proposition}[chapter]
\newtheorem{corollary}{Corollary}[chapter]
\theoremstyle{remark}
\newtheorem{observation}{Observation}[chapter]
\newtheorem{terminology}{Terminology}[chapter]
\newtheorem{remark}{Remark}[chapter]    
\newtheorem{example}{Example}[chapter]

\setcounter{tocdepth}{3}

% Begin document


\begin{document}
\begin{center}
\begin{tikzpicture}[remember picture,overlay]
\node[anchor=center, inner sep= 0pt] 
{\includegraphics[width=.5\textwidth]
{logo_centenario-eps-converted-to.pdf}};
\centering
\end{tikzpicture}
\vfill
\LARGE
\textsc{Università degli Studi di Trieste}\\
\rule{\textwidth}{0.1mm}\\
\large
\textsc{Dipartimento di Matematica Informatica e Geoscienze -- MIGE}\\ 
\bigskip
Corso di Laurea in Intelligenza Artitificiale e Data Analytics \\
\vfill %%\vspace{3cm}
\textsc{Tesi di laurea triennale}\\
\vfill %%\vspace{3cm}
\LARGE
\textbf{\textsc{TITOLO}}\\
\end{center}
\vfill
%%%%%%
%%%%%% Qui va inserito il nome del laureando
\begin{tabular}[t]{l}
Laureando:\\
\textbf{Samuele Serri}\\
\textsc{Matricola: SM3201369}
\end{tabular}
%%%%%%
\hfill 
%%%%%% Qui va inserito il nome del relatore
\begin{tabular}[t]{l}
Relatore: \\
\textbf{Prof. Nome Cognome} \\
\\
Correlatore: \\
\textbf{Prof. Nome Cognome}
\end{tabular}
%%%%%%
\vfill
\begin{center}
\normalsize
\rule{8cm}{0.1mm}\\
\bigskip
ANNO ACCADEMICO 2024--2025
\end{center}
\thispagestyle{empty}  


\clearpage              
\pagenumbering{roman}
\shipout\null 
\tableofcontents
\bigskip
% pagina pdflatex -interaction=nonstopmode main.texvuota


\clearpage
% notation page not numbered
\thispagestyle{empty}
\chapter*{Notation}
\begin{itemize}
    \item[--] $\R$: Set of real numbers.
    \item[--] $\R_+$: Set of non-negative real numbers.
    \item[--] $\R^{m \times n}$: Set of real matrices with $m$ rows and $n$ columns.
    \item[--] $\R_+^{m \times n}$: Set of non-negative real matrices with $m$ rows and $n$ columns.
    \item[--] $A[i,j]$: Element of matrix $A$ at row $i$ and column $j$.
    \item[--] $A[i,:]$: $i$-th row of matrix $A$.
    \item[--] $A[:,j]$: $j$-th column of matrix $A$.
    \item[--] $\|A\|_F$: Frobenius norm of matrix $A$. Defined as $\|A\|_F = \sqrt{\sum_{i=1}^{m} \sum_{j=1}^{n} A[i,j]^2}$.
    \item[--] $A^\top$: Transpose of matrix $A$.
    \item[--] $A^\dagger$: Pseudoinverse of matrix $A$.% \footnote{https://en.wikipedia.org/wiki/Moore-Penrose\_inverse}
    \item[--] $A \geq 0$: Matrix $A$ is non-negative, i.e., all its elements are non-negative.
    \item[--] $A > 0$: Matrix $A$ is positive, i.e., all its elements are strictly positive.
    \item[--] $A \circ B$: Element-wise product of matrices $A$ and $B$.
    \item[--] $\| A \|$: Norm of matrix $A$.
    \item[--] $\| A \|_F$: Frobenius norm of matrix $A$.
    \item[--] $\nabla f$: Gradient of function $f$.
\end{itemize}
\clearpage
\pagenumbering{arabic}
\include{chapters/Introduction}
\chapter{Background}
This chapter contains all the preliminaries necessary to formally define the NMF problem and understand the algorithms presented in this work.
\section{Linear algebra}
Basic compcet of linear algebra including matrices, vectors, and operations on them.
\begin{definition}[Matrix]
    Given two numbers $m,n \in N$, a matrix $A \in \R^{m \times n}$ is a rectangular array of real numbers with $m$ rows and $n$ columns.
\end{definition}
\begin{equation*}
    A = \begin{bmatrix}
    a_{11} & a_{12} & \cdots & a_{1n} \\
    a_{21} & a_{22} & \cdots & a_{2n} \\
    \vdots & \vdots & \ddots & \vdots \\
    a_{m1} & a_{m2} & \cdots & a_{mn}
    \end{bmatrix}
\end{equation*}
\begin{definition}[transpose of a matrix]
    Given a matrix $A \in \R^{m \times n}$, the transpose of $A$, denoted by $A^\top$, is the matrix obtained by swapping the rows and columns of $A$.
    That is, the matrix $A^\top$ is a matrix such that $A[i,j] = A^\top[j,i]$ for all $1 \leq i \leq m$ and $1 \leq j \leq n$.
\end{definition}
\begin{observation}
    The transpose of a matrix $A \in \R^{m \times n}$ is a matrix $A^\top \in \R^{n \times m}$.
\end{observation}
\begin{definition}[Product of matrices]
    Given two matrices $A \in \R^{m \times n}$ and $B \in \R^{n \times p}$, the product of $A$ and $B$, denoted by $AB$, is the matrix $C \in \R^{m \times p}$ such that:
    \begin{equation*}
    C[i,j] = \sum_{k=1}^{n} A[i,k] B[k,j]
    \end{equation*}
    for all $1 \leq i \leq m$ and $1 \leq j \leq p$.
\end{definition}
\begin{definition}[Column vector]
    A column vector $x \in \R^m$ is a matrix with $m$ rows and 1 column. (i.e. $x \in \R^{m \times 1}$)
\end{definition}
\begin{terminology}
    In this text the word vector will always refer to a column vector unless otherwise specified.
\end{terminology}
\begin{equation*}
    x = \begin{bmatrix}
    x_{1} \\
    x_{2} \\
    \vdots \\
    x_{m}
    \end{bmatrix}
\end{equation*}
\begin{definition}[Row vector]
    A row vector $y \in \R^n$ is the transpose of a column vector. (i.e. $y \in \R^{1 \times n}$)
\end{definition}
\begin{equation*}
    y =\begin{bmatrix}
    y_1 \\
    y_2 \\
    \vdots \\
    y_n
    \end{bmatrix}^\top =\begin{bmatrix}
    y_{1} & y_{2} & \cdots & y_{n}
    \end{bmatrix}
\end{equation*}
\subsection{Norms}
\begin{definition}[Norm of a vector]
    Given a vector $x \in \R^n$, a norm is a function $\| \cdot \|: \R^n \rightarrow \R_+$ that satisfies the following properties:
    \begin{itemize}
        \item $\| x \| \geq 0$ for all $x \in \R^n$ and $\| x \| = 0$ if and only if $x = 0$.
        \item $\| \alpha x \| = |\alpha| \| x \|$ for all $\alpha \in \R$ and $x \in \R^n$.
        \item $\| x + y \| \leq \| x \| + \| y \|$ for all $x, y \in \R^n$ (triangle inequality).
    \end{itemize}
\end{definition}
\begin{example}
    The Euclidean norm (or 2-norm) of a vector $x \in \R^n$ is defined as:
    
    \begin{equation}
    \| x \|_2 = \sqrt{\sum_{i=1}^{n} x_i^2}
    \end{equation}
    The 1-norm of a vector $x \in \R^n$ is defined as:
    \begin{equation}
    \| x \|_1 = \sum_{i=1}^{n} |x_i|
    \end{equation}
    The infinity norm of a vector $x \in \R^n$ is defined as:
    \begin{equation}
    \| x \|_\infty = \max_{1 \leq i \leq n} |x_i|
    \end{equation}
\end{example}
\begin{definition}[Norm of a matrix]
    Given a matrix $A \in \R^{m \times n}$, a norm is a function $\| \cdot \|: \R^{m \times n} \rightarrow \R_+$ that satisfies the following properties:
    \begin{itemize}
        \item $\| A \| \geq 0$ for all $A \in \R^{m \times n}$ and $\| A \| = 0$ if and only if $A = 0$.
        \item $\| \alpha A \| = |\alpha| \| A \|$ for all $\alpha \in \R$ and $A \in \R^{m \times n}$.
        \item $\| A + B \| \leq \| A \| + \| B \|$ for all $A, B \in \R^{m \times n}$ (triangle inequality).
    \end{itemize}
\end{definition}
\begin{example}
    The Frobenius norm of a matrix $A \in \R^{m \times n}$ is defined as:
    \begin{equation}
    \| A \|_F = \sqrt{\sum_{i=1}^{m} \sum_{j=1}^{n} a_{ij}^2}
    \end{equation}
    The 1-norm of a matrix $A \in \R^{m \times n}$ is defined as:
    \begin{equation}
    \| A \|_1 = \max_{1 \leq j \leq n} \sum_{i=1}^{m} |a_{ij}|
    \end{equation}
    The infinity norm of a matrix $A \in \R^{m \times n}$ is defined as:
    \begin{equation}
    \| A \|_\infty = \max_{1 \leq i \leq m} \sum_{j=1}^{n} |a_{ij}|
    \end{equation}
\end{example}
\newpage
\section{Optimization}
Basic concepts of optimization including convexity, gradient descent, and other optimization algorithms.
\begin{definition}
    An optimization problem is defined as:
    \begin{equation}
        \min_{x \in X}f(x)
    \end{equation}
    where \begin{itemize}
        \item $X$ is the feasible set in which the solution must lie.
        \item $f: X \rightarrow \R$ is the objective function to be minimized
    \end{itemize}
\end{definition}
\begin{terminology}
    In other context the objective function is also referred to as cost function, loss function, or energy function. Here will use them interchangeably.
\end{terminology}
\begin{definition}[Global optimum]
    A point $x^* \in X$ is a global optimum of the optimization problem $\min_{x \in X}f(x)$ if:
    \begin{equation}
        f(x^*) \leq f(x) \quad \forall x \in X
    \end{equation}
\end{definition}
\begin{definition}[Local optimum]
    A point $x^* \in X$ is a local optimum of the optimization problem $\min_{x \in X}f(x)$ if there exists a neighborhood $U$ of $x^*$ such that:
    \begin{equation}
        f(x^*) \leq f(x) \quad \forall x \in U \cap X
    \end{equation}
\end{definition}
\begin{example}[Linear least squares problem]
    Given a matrix $A \in \R^{m \times n}$ and a vector $b \in \R^m$, the linear least squares problem is defined as:
    \begin{equation}
        \min_{x \in \R^n} \| Ax - b \|_2^2
    \end{equation}  
\end{example}
\begin{theorem}[solution of the linear least squares problem]
    The solution of the linear least squares problem $\min_{x \in \R^n} \| Ax - b \|_2^2$ is given by:
    \begin{equation}
        x^* = (A^\top A)^{-1} A^\top b = A^\dagger A^\top b
    \end{equation}
    provided that $A^\top A$ is invertible.
\end{theorem}
\begin{proof}
    The objective function can be rewritten as:
    \begin{equation*}
        f(x) = \| Ax - b \|_2^2 = (Ax - b)^\top (Ax - b) = x^\top A^\top A x - 2b^\top A x + b^\top b
    \end{equation*}
    To find the minimum, we compute the gradient of $f(x)$ and set it to zero:
    \begin{equation*}
        \nabla f(x) = 2A^\top A x - 2A^\top b
    \end{equation*}
    Setting the gradient to zero gives:
    \begin{equation*}
        2A^\top A x - 2A^\top b = 0
    \end{equation*}
    which simplifies to:
    \begin{equation*}
        A^\top A x = A^\top b
    \end{equation*}
    Assuming that $A^\top A$ is invertible, we can solve for $x$:
    \begin{equation*}
        x^* = (A^\top A)^{-1} A^\top b
    \end{equation*}
    This completes the proof.
\end{proof}
% illustration of least squares
\begin{figure}[h]
    \centering
    \includegraphics[width=0.5\textwidth]{Imgs/Least_squares.png}
    \caption{In a three-dimensional setting
    the least squares regression line becomes a plane. The plane is chosen
to minimize the sum of the squared vertical distances between each vector
(shown in red) and the plane.
\\  \vspace{2pt} \small\textit{Image from: \cite{IntroToSL}}}
    \label{fig:least_squares}
\end{figure}
\newpage
\subsection{Gradient descent}\label{Gradient descent section}
\begin{definition}[Gradient descent]
    Gradient descent is an iterative optimization algorithm used to find a local minimum of a differentiable function. Given an initial point $x_0 \in X$, the algorithm updates the point iteratively using the following rule:
    \begin{equation}
        x_{k+1} = x_k - \eta_k \nabla f(x_k)
    \end{equation}
    where $\eta_k > 0$ is the step size (or learning rate) at iteration $k$ and $\nabla f(x_k)$ is the gradient of the objective function at point $x_k$.
\end{definition}
\begin{figure}[h]
    \centering
    \includegraphics[width=0.5\textwidth]{Imgs/Gradient_descent.png}
    \caption{Illustration of gradient descent on a series of level sets.%
    \\ \vspace{2pt} \small\textit{Source: Wikimedia Commons -- see \url{https://commons.wikimedia.org/wiki/File:Gradient_descent.png} for author and license information.}}
    \label{fig:gradient_descent}
\end{figure}
\par
We have the following convergence result for gradient descent:

\newpage
\section{Linear dimensionality reduction}
Basic concepts of linear dimensionality reduction including PCA, SVD, and other related techniques.
\chapter{Non-Negative Matrix Factorization}
This chapter introduces the NMF problem.
\section{Definition}
In its more general form the Non-negative matrix factorization problem can be defined as follows: given a non-negative (all the elements must be non-negative) matrix $V \in \R_+^{m \times n}$ and a factorization rank $r \in \N$ find two matrices
$W \in \R_+^{m \times r}$ and $H \in \R_+^{r \times n}$, constrained to be non-negative, such that their product $WH$ approximates $V$ as closely as possible according to some distance measure $D$.
\par
This problem can be formulated with an optimization problem as follows:
\begin{definition}[Non-Negative Matrix Factorization optimization problem]\label{NMF optimization problem}
Given a non-negative matrix $V \in \R_+^{m \times n}$ and a factorization rank $r \in \N$, find two non-negative matrices $W \in \R_+^{m \times r}$ and $H \in \R_+^{r \times n}$ that minimize 
a given cost function $D(V, WH)$, that is:
\begin{equation}
\min_{W \in \R_+^{m \times r}, H \in \R_+^{r \times n}} D(V, WH)
\end{equation}
where $D$ is a distance measure between matrices.
\end{definition}

PARLARE DEI PROBLEMI (NP-HARD, NON UNICITÀ DELLA SOLUZIONE)\\
PARLARE DELLE DIVERSE DISTANZE USATE COMUNEMENTE (FROBENIUS, KL, ETC) E DEL PERCHÈ SI USANO.\\
ESEMPI DI APPLICAZIONI\\
INTERPRETAZIONE GEOMETRICA\\

\section{Historical origin of NMF}
Even though some early works date back to the 70s in the field of Earth science and remote system sensing \cite{Wallace1960}\cite{Imbrie1964}\cite{craig1994}
the first modern definition of the NMF problem is usually attributed to the work of Paatero and Tapper in 1994 \cite{Paatero1994} where they defined the so-called Positive Matrix Factorization (PMF) problem, a particular case of NMF where the distance measure used is the least squares one (Frobenius norm).
This early work arose in the field of analytical chemistry and the NMF model as a modern data analysis tool remained relatively unknown until the seminal paper of Lee and Seung in 1999 \cite{Lee1999}.
\par
In this work Lee and Seung introduces the NMF model to the machine learning community and proposed an easy to implement algorithm based on multiplicative updates to solve the NMF optimization problem.
Their work is especially famous because they  demonstrate an algorithm for non-negative matrix factorization that is able to learn parts of faces, in contrast to other decomposition methods like PCA or SVD that learn holistic features.
Their results are also reproduced in this thesis to show an example of application of NMF.
\section{Two applications of NMF}
In this section two applications of NMF are presented: the first one is the well-known example of learning parts of faces introduced by Lee and Seung in \cite{Lee1999},
while the second one is an application of NMF to text mining, in particular to topic modeling.
\subsection{Learning parts of faces}
TODO
\subsection{Topic modeling}
TODO
\chapter{Algorithms for NMF}
In this chapter, some algorithms are derived and presented.\
\par
Recall that the NMF problem \ref{NMF optimization problem} is NP-hard \cite{vavasis2007complexitynonnegativematrixfactorization}, hence all the Algorithms presented in this chapter are heuristics that aim to find a local minimum of the optimization problem.
More precisely, they are iterative schemes that start from an initial guess of the factors \( W \) and \( H \) and iteratively update them to reduce the value of the objective function, converging to a local minimum.
They don't guarantee to converge to the global minimum.
\section{Two blocks coordinate descent (2-BCD)}
The most common setting for NMF is the so called two-blocks coordinate descent, in this setting the factors \( W \) and \( H \) are updated alternately, fixing one while updating the other.

\begin{algorithm}[h]\label{2-BCD}
    \caption{Two-blocks coordinate descent for NMF}
    \begin{algorithmic}[1]
        \Require \( V \in \mathbb{R}_+^{m \times n} \), rank \( r \), initial factors \( W^{(0)} \in \mathbb{R}_+^{m \times r} \), \( H^{(0)} \in \mathbb{R}_+^{r \times n} \)
        % \output \( W \in \mathbb{R}_+^{m \times r} \), \( H \in \R_+^{r \times n} \) such that \( V \approx WH \)
        \While{stopping criterion not met}
            \State \(W^{(k)}\) $\gets$ Update (\( W^{(k - 1)} \), \( H^{(k-1)} \))
            \Comment{Update \( W \) by fixing \( H \)}
            \State \(H^{(k)}\) $\gets$ Update (\( H^{(k - 1)} \), \( W^{(k)} \))
            \Comment{Update \( H \) by fixing \( W \)}
        \EndWhile
        \State \Return \( W, H \)
    \end{algorithmic}
\end{algorithm}
This general scheme is widely used, and we will present different ways to update the factors \( W \) and \( H \). There are mainly two observations that justify this approach:
\begin{enumerate}
    \item When one of the factors is fixed, the optimization subproblem is convex for most of the common cost functions. That makes it easier to design algorithms to solve them.
    \item Due to the symmetry of the problem $V = WH \iff V^T = H^T W^T$ the value of the cost function is the same if we swap \( W \) and \( H \) and transpose the input matrix \( V \). $D(V, WH) = D(V^T, H^T W^T)$.
    Therefore, it is sufficient to design an update rule for one of the factors, the other can be obtained by swapping the roles of \( W \) and \( H \) and transposing \( V \).
\end{enumerate}
Overall this approach is much easier to handle than trying to update both factors simultaneously. See \cite{mukkamala2019alternatingupdatesmatrixfactorization}.
\subsection{Exact 2-BCD}
If the subproblems of \ref{2-BCD} are solved exactly at each iteration, the method is called exact 2-BCD. 
And takes the following form:
\begin{algorithm}[H]\label{Exact 2-BCD}
    \caption{Exact two-blocks coordinate descent for NMF}
    \begin{algorithmic}[1]
        \Require \( V \in \mathbb{R}_+^{m \times n} \), rank \( r \), initial factors \( W^{(0)} \in \mathbb{R}_+^{m \times r} \), \( H^{(0)} \in \mathbb{R}_+^{r \times n} \)
        % \output \( W \in \mathbb{R}_+^{m \times r} \), \( H \in \R_+^{r \times n} \) such that \( V \approx WH \)
        \While{stopping criterion not met}
            \State \( W^{(k)} \leftarrow \arg\min_{W \geq 0} D(V, WH^{(k - 1)}) \)
            \State \( H^{(k)} \leftarrow \arg\min_{H \geq 0} D(V, W^{(k)}H) \)
        \EndWhile
        \State \Return \( W, H \)
    \end{algorithmic}
\end{algorithm}
In this case we have the following convergence result:
\begin{theorem}
    Let $\min_{W,H \geq 0} D(V, WH)$ be the NMF optimization problem, and
    \( \{(W^{(k)}, H^{(k)})\}_{k \in \mathbb{N}} \) be the sequence generated by the exact 2-BCD method.
    if \begin{enumerate}
        \item The function \( D \) is continuously differentiable.
        \item Each block of variables $W$ and $H$ belong to a closed convex set. \label{Condition 2 Exact 2-BCD}
    \end{enumerate}
    Then the limit points of the sequence \( \{(W^{(k)}, H^{(k)})\}_{k \in \mathbb{N}} \) are stationary points.
\end{theorem}
\begin{proof}
    this theorem is a corollary (Corollary 2) of the main theorem presented in \cite{GRIPPO2000127}.
\end{proof}
\begin{observation}
    In the NMF problem, the condition \ref{Condition 2 Exact 2-BCD} is always satisfied because the elements of $W$ and $H$ are constrained to lie in the non negative orthant.
\end{observation}
\section{Stopping criteria}
All the algorithms presented in this chapter are iterative methods that require a stopping criterion to terminate.
Usually, the implementation of these algorithms include a maximum number of iterations after which the algorithm stops.
However, others stopping criteria can be used in addition to the maximum number of iterations.
\par
A common stopping criterion is to check the relative error at each step and stop the algorithm if no significant improvement is observed for a given number of steps $T$.
For instance, we can use the following rule:
\begin{equation}
    |e(t - T) - e(t)| \leq tol\times e(t)
\end{equation}
Where $e(t)$ is the relative error at iteration $t$. This is equal to:
\begin{equation}  \label{stopping criterion I}
    e(t) = \frac{\|V - W^{(t)}H^{(t)}\|_F^2}{\|V\|_F^2}
\end{equation}
in the case of the Frobenius norm. 
$tol$ is a small tolerance value,  for instance $10^{-4}$, and $T$ is the lag used to compare the current error with the error $T$ iterates before.
\par
Another effective stopping criterion is to check the norm of the difference between the iterates and stop if it falls below a certain threshold.
\begin{align}
    &\|W^{(t)} - W^{(t-T)}\|_F^2 \leq tol \times \|W^{(t - T)}\|_F^2 \\
    &\|H^{(t)} - H^{(t-T)}\|_F^2 \leq tol \times \|H^{(t - T)}\|_F^2
\end{align}
Again a small value for $tol$, for instance $10^{-4}$, and $T = 10$ are reasonable choices in practice.
\par
The complete algorithm with the stopping criterion \ref{stopping criterion I} is summarized in the following pseudo-code:
\begin{algorithm}[H]\label{2-BCD}
    \caption{Two-blocks coordinate descent for NMF with stopping criteria \ref{stopping criterion I}}
    \begin{algorithmic}[1]
        \Require \( V \in \mathbb{R}_+^{m \times n} \), rank \( r \), initial factors \( W^{(0)} \in \mathbb{R}_+^{m \times r} \), \( H^{(0)} \in \mathbb{R}_+^{r \times n} \), tolerance \( tol \), lag \( T \), maximum iterations \( \text{max\_iter} \)
        % \output \( W \in \mathbb{R}_+^{m \times r} \), \( H \in \R_+^{r \times n} \) such that \( V \approx WH \)
        \State Compute \( e(0) = \frac{\|V - W^{(0)}H^{(0)}\|_F^2}{\|V\|_F^2} \)
        \Comment{Initial relative error}
        \For {\( k = 0, 1, 2, \ldots, \text{max\_iter} \)}
            \State \(W^{(k)}\) $\gets$ Update (\( W^{(k - 1)} \), \( H^{(k-1)} \))
            \Comment{Update \( W \) by fixing \( H \)}
            \State \(H^{(k)}\) $\gets$ Update (\( H^{(k - 1)} \), \( W^{(k)} \))
            \Comment{Update \( H \) by fixing \( W \)}
            \State Compute \( e(k) = \frac{\|V - W^{(k)}H^{(k)}\|_F^2}{\|V\|_F^2} \)
            \Comment{Compute relative error}
            \If {\( k  \geq T \) \textbf{and} \( |e(k - T) - e(k)| \leq tol \times e(k) \)}
                \State \textbf{break}
                \Comment{If stopping criterion met, exit loop}
            \EndIf
        \EndFor
        \State \Return \( W, H \)
    \end{algorithmic}
\end{algorithm}
\section{Initialization}
TODO
\section{Multiplicative Updates (MU)}
The multiplicative updates (MU) were first introduced by Lee and Seung in \cite{Lee1999}. They are one of the most popular algorithms for NMF.
In this section we will derive the MU algorithm for two different cost functions: the Frobenius norm and the Kullback-Leibler divergence.
The Lee and Seung paper with converge proof is \cite{lee2000algorithms}. A derivation using matrix algebra can be found in \cite{burred2014detailed}.
In section \ref{MU beta section} we will present the MU algorithm for the more general case of beta divergence \cite{févotte2011algorithmsnonnegativematrixfactorization}.
\subsection{MU with the Frobenius norm}
In this section we will derive the MU iteration scheme for the case where
$D(V, WH) = \|V - WH\|_F^2$ is the Frobenius norm.
The scheme can be interpreted as a scaled gradient descent method.\\
\par
The gradient of the cost function with respect to \( H \) is given by:
\begin{equation}
    \nabla_H \frac{1}{2} \|V - WH\|_F^2 = W^\top (WH - V) = W^\top WH - W^\top V
\end{equation}
A re-scaled gradient descent update for \( H \) is given by:
\begin{equation}
    H \leftarrow H - \frac{H}{W^\top WH} \circ \nabla_H \frac{1}{2} \|V - WH\|_F^2
\end{equation}
This leads to the following update rule:
\begin{equation}
    H \leftarrow H \circ \frac{W^\top V}{W^\top WH}
\end{equation}
Similarly, the update rule for \( W \) can be derived by fixing \( H \)
\begin{equation}
    \nabla_W \frac{1}{2} \|V - WH\|_F^2 = (WH - V) H^\top = W H H^\top - V H^\top
\end{equation}
A re-scaled gradient descent update for \( W \) is given by:
\begin{equation}
    W \leftarrow W - \frac{W}{W H H^\top} \circ \nabla_W \frac{1}{2} \|V - WH\|_F^2
\end{equation}
This leads to the following update rule:
\begin{equation}
    W \leftarrow W \circ \frac{V H^\top}{W H H^\top}
\end{equation}
\par
Recall that in section \ref{Gradient descent section} we have introduce the gradient descent as an iterative method for finding local minima with the following update rule: $x_{k+1} = x_k - \eta_k\nabla f(x_k)$.
The re-scaled gradient descent used in the MU updates can be interpreted as a gradient descent with a variable step size $\eta_k$ that is constrained to ensure the non-negativity of the iterates.
We derive the MU updates for $W$:
\begin{equation} \label{GD MU updates}
    W^{(k+1)} = W^{(k)} - \eta_k \nabla_W \frac{1}{2} \|V - W^{(k)}H\|_F^2
\end{equation}
To ensure the non-negativity of the iterates, we impose the following condition:
\begin{equation}
    W^{(k+1)} = W^{(k)} - \eta_k \nabla_W \frac{1}{2} \|V - W^{(k)}H\|_F^2 \geq 0
\end{equation}
This leads to the following inequality:
\begin{equation}
    \eta_k \leq \frac{W^{(k)}}{\nabla_W \frac{1}{2} \|V - W^{(k)}H\|_F^2} = \frac{W^{(k)}}{W^{(k)} H H^\top - V H^\top}
\end{equation}
Since $VH^\top \geq 0$ and $W^{(k)} H H^\top \geq 0$, we can write the following inequality:
\begin{equation}
    \eta_k \leq \frac{W^{(k)}}{W^{(k)} H H^\top} < \frac{W^{(k)}}{W^{(k)} H H^\top - V H^\top}
\end{equation}
The selecting the step size as:
\begin{equation}
    \eta_k = \frac{W^{(k)}}{W^{(k)} H H^\top}
\end{equation}
and substituting in equation \ref{GD MU updates} we obtain the MU update for \( W \):
\begin{equation}
    W^{(k+1)} = W^{(k)} - \frac{W^{(k)}}{W^{(k)} H H^\top} \circ (W^{(k)} H H^\top - V H^\top)
\end{equation}
And simplifying we obtain the update rule for \( W \):
\begin{equation}
    \boxed{W^{(k+1)} = W^{(k)} \circ \frac{V H^\top}{W^{(k)} H H^\top}}
\end{equation}
Analogously, we can derive the MU update for \( H \) and using the gradient $\nabla_H \frac{1}{2} \|V - WH\|_F^2 = W^\top WH - W^\top V$ we obtain:
\begin{equation}
    \boxed{H^{(k+1)} = H^{(k)} \circ \frac{W^\top V}{W^\top W H^{(k)}}}
\end{equation}
\par The complete MU algorithm is summarized in the following algorithm:
\begin{algorithm}[H]\label{MU for Frobenius}
    \caption{Multiplicative Updates for NMF with the Frobenius norm}
    \begin{algorithmic}[1]
        \Require \( V \in \mathbb{R}_+^{m \times n} \), rank \( r \), initial factors \( W^{(0)} \in \mathbb{R}_+^{m \times r} \), \( H^{(0)} \in \mathbb{R}_+^{r \times n} \)
        % \output \( W \in \mathbb{R}_+^{m \times r} \), \( H \in \R_+^{r \times n} \) such that \( V \approx WH \)
        \While{stopping criterion not met}
            \State \( H^{(k+1)} \leftarrow H^{(k)} \circ \frac{W^{(k) \top} V}{W^{(k) \top} W^{(k)} H^{(k)}} \)
            \State \( W^{(k+1)} \leftarrow W^{(k)} \circ \frac{V H^{(k+1) \top}}{W^{(k)} H^{(k+1)} H^{(k+1) \top}} \)
        \EndWhile
        \State \Return \( W, H \)
    \end{algorithmic}
\end{algorithm}
\subsubsection{Convergence of MU with the Frobenius norm}
\begin{theorem}[\cite{lee2000algorithms}, Theorem 1] \label{MU convergence Fro}
    The value of the objective function $D(V, WH) = \|V - WH \|_F^2$ is non-increasing when $W^{(k)}, H^{(k)}$ are produced by Multiplicative updates \ref{MU for Frobenius}.
    Furthermore $W^{(k)}$ and $H^{(k)}$ are stationary points of the objective $D(V, WH)$ if and only if the distance is invariant under the iterative scheme \ref{MU for Frobenius}.
\end{theorem}
\begin{proof}
    TODO
\end{proof}
\subsubsection{Computational cost}
The computational cost of the MU algorithm with the Frobenius norm depends on the computation of the gradients, $WHH^\top$ and $VH^\top$ for the update of $W$.
And $W^\top WH$ and $W^\top V$ for the update of $H$.
Recall that the dimensions of the matrices are:
\begin{itemize}
    \item \( V \in \mathbb{R}_+^{m \times n} \)
    \item \( W \in \mathbb{R}_+^{m \times r} \)
    \item \( H \in \mathbb{R}_+^{r \times n} \)
\end{itemize}
When computing $WHH^\top$ one should pay attention to compute $W(HH^\top)$ instead of $(WH)H^\top$ to reduce the computational cost.
The cost of the latter is $O(mnr + mnr^2)$ while the cost of the former is $O(mr^2 + r^2 n) = O(r^2(m+n))$. And in most NMF applications $r \ll \min(m,n)$.
Similarly, when computing $W^\top WH$ one should compute $(W^\top W)H$ instead of $W^\top (WH)$ to reduce the computational cost. Again we have that the former requires $O(mr^2 + r^2 n) = O(r^2(m+n))$ operations while the latter requires $O(mnr + mnr^2)$ operations.
\subsection{MU with the Kullback-Leibler divergence}
In this section we will derive the MU iteration scheme for the case where
$D(V, WH) = D_{KL}(V \| WH)$ is the Kullback-Leibler divergence. 
Again this scheme can be interpreted as a scaled gradient descent method, in fact more generally the MU updates can be interpreted as a scaled gradient descent method for a wide class of cost functions.
\par
The gradient of the cost function with respect to \( H \) is given by:
\begin{equation}
    \nabla_H D_{KL}(V \| WH) = W^\top \left(\textbf{1} - \frac{V}{WH} \right) = W^\top \textbf{1} - W^\top \frac{V}{WH}
\end{equation}
Where $\textbf{1}$ is the matrix of all ones with the same dimensions as \( V \).
A re-scaled gradient descent update for \( H \) is given by:
\begin{equation}
    H \leftarrow H - \frac{H}{W^\top \textbf{1}} \circ \nabla_H D_{KL}(V \| WH)
\end{equation}
This leads to the following update rule:
\begin{equation}
    H \leftarrow H \circ \frac{W^\top \frac{V}{WH}}{W^\top \textbf{1}}
\end{equation}
Similarly, the update rule for \( W \) can be derived by fixing \( H \)
\begin{equation}
    \nabla_W D_{KL}(V \| WH) = \left(\textbf{1} - \frac{V}{WH} \right) H^\top = \textbf{1} H^\top - \frac{V}{WH} H^\top
\end{equation}
A re-scaled gradient descent update for \( W \) is given by:
\begin{equation}
    W \leftarrow W - \frac{W}{\textbf{1} H^\top} \circ \nabla_W D_{KL}(V \| WH)
\end{equation}
This leads to the following update rule:
\begin{equation}
    W \leftarrow W \circ \frac{\frac{V}{WH} H^\top}{\textbf{1} H^\top}
\end{equation}
Once again using the same reasoning as in the previous section, we can interpret the re-scaled gradient descent used in the MU updates as a gradient descent with a variable step size $\eta_k$ that is constrained to ensure the non-negativity of the iterates.
We derive the MU updates for $W$:
\begin{equation} \label{GD MU updates KL}
    W^{(k+1)} = W^{(k)} - \eta_k \nabla_W D_{KL}(V \| W^{(k)}H)
\end{equation}
To ensure the non-negativity of the iterates, we impose the following condition:
\begin{equation}
    W^{(k+1)} = W^{(k)} - \eta_k \nabla_W D_{KL}(V \| W^{(k)}H) \geq 0
\end{equation}
This leads to the following inequality:
\begin{equation}
    \eta_k \leq \frac{W^{(k)}}{\nabla_W D_{KL}(V \| W^{(k)}H)} = \frac{W^{(k)}}{\textbf{1} H^\top - \frac{V}{W^{(k)}H} H^\top}
\end{equation}
Since $\frac{V}{W^{(k)}H} H^\top \geq 0$ and $\textbf{1} H^\top \geq 0$, we can write the following inequality:
\begin{equation}
    \eta_k \leq \frac{W^{(k)}}{\textbf{1} H^\top} < \frac{W^{(k)}}{\textbf{1} H^\top - \frac{V}{W^{(k)}H} H^\top}
\end{equation}
The selecting the step size as:
\begin{equation}
    \eta_k = \frac{W^{(k)}}{\textbf{1} H^\top}
\end{equation}
and substituting in equation \ref{GD MU updates KL} we obtain the MU update for \( W \):
\begin{equation}
    W^{(k+1)} = W^{(k)} - \frac{W^{(k)}}{\textbf{1} H^\top} \circ (\textbf{1} H^\top - \frac{V}{W^{(k)}H} H^\top)
\end{equation}
And simplifying we obtain the update rule for \( W \):
\begin{equation}
    \boxed{W^{(k+1)} = W^{(k)} \circ \frac{\frac{V}{W^{(k)}H} H^\top}{\textbf{1} H^\top}}
\end{equation}
Analogously, we can derive the MU update for \( H \) and using the gradient $\nabla_H D_{KL}(V \| WH) = W^\top \textbf{1} - W^\top \frac{V}{WH}$ we obtain:
\begin{equation}
    \boxed{H^{(k+1)} = H^{(k)} \circ \frac{W^{(k) \top} \frac{V}{W^{(k)} H^{(k)}}}{W^{(k) \top} \textbf{1}}}
\end{equation}
\par The complete MU algorithm is summarized in the following algorithm:
\begin{algorithm}[H]\label{MU for KL}
    \caption{Multiplicative Updates for NMF with the Kullback-Leibler divergence}
    \begin{algorithmic}[1]
        \Require \( V \in \mathbb{R}_+^{m \times n} \), rank \( r \), initial factors \( W^{(0)} \in \mathbb{R}_+^{m \times r} \), \( H^{(0)} \in \mathbb{R}_+^{r \times n} \)
        % \output \( W \in \mathbb{R}_+^{m \times r} \), \( H \in \R_+^{r \times n} \) such that \( V \approx WH \)
        \While{stopping criterion not met}
            \State \( H^{(k+1)} \leftarrow H^{(k)} \circ \frac{W^{(k) \top} \frac{V}{W^{(k)} H^{(k)}}}{W^{(k) \top} 1} \)
            \State \( W^{(k+1)} \leftarrow W^{(k)} \circ \frac{\frac{V}{W^{(k)} H^{(k+1)}} H^{(k+1) \top}}{1 H^{(k+1) \top}} \)
        \EndWhile
        \State \Return \( W, H \)
    \end{algorithmic}
\end{algorithm}
\subsubsection{Convergence of MU with the Kullback-Leibler divergence}
\begin{theorem}[\cite{lee2000algorithms}, Theorem 2] \label{MU convergence KL}
    The value of the objective function $D(V, WH) = D_{KL}(V \| WH)$ is non-increasing when $W^{(k)}, H^{(k)}$ are produced by Multiplicative updates \ref{MU for KL}.
    Furthermore $W^{(k)}$ and $H^{(k)}$ are stationary points of the objective $D(V, WH)$ if and only if the distance is invariant under the iterative scheme \ref{MU for KL}.
\end{theorem}
\begin{proof}
    TODO
\end{proof}
\subsubsection{Computational cost}
\section{Alternating Least Squares (ALS)}
TODO
\section{Hierarchical Alternating Least Squares (HALS)}
TODO
\section{MU with beta divergence} \label{MU beta section}
The $\beta$-divergence is a family of cost functions with a single parameter $\beta$ that generalizes several common cost functions used in NMF, including the Frobenius norm and the Kullback-Leibler divergence.
For the definition of the beta divergence see section \ref{Beta divergence section}.
\par
In this section we will derive the MU iteration scheme for $D(V, WH) = D_{\beta}(V, WH)$ the beta divergence in the general case.
One can see that the previous results for the Frobenius norm and the Kullback-Leibler divergence are special cases of the beta divergence for $\beta = 2$ and $\beta = 1$ respectively.
However, since they are the most popular cost functions used in NMF, we presented them separately.
Furthermore very few theoretical results are known for values of $\beta$ other than 0, 1 and 2.
\par
The gradient of the cost function with respect to \( H \) is given by:
\begin{equation}
    \nabla_H D_{\beta}(V, WH) = W^\top \left( (WH)^{\beta - 1} - V \circ (WH)^{\beta - 2} \right)
\end{equation}
Where the exponentiation is intended element-wise.
A re-scaled gradient descent update for \( H \) is given by:
\begin{equation}
    H \leftarrow H - \frac{H}{W^\top (WH)^{\beta - 1}} \circ \nabla_H D_{\beta}(V, WH)
\end{equation}
This leads to the following update rule:
\begin{equation}
    H \leftarrow H \circ \frac{W^\top (V \circ (WH)^{\beta - 2})}{W^\top (WH)^{\beta - 1}}
\end{equation}
Similarly, the update rule for \( W \) can be derived by fixing \( H \)
\begin{equation}
    \nabla_W D_{\beta}(V, WH) = \left( (WH)^{\beta - 1} - V \circ (WH)^{\beta - 2} \right) H^\top
\end{equation}
A re-scaled gradient descent update for \( W \) is given by:
\begin{equation}
    W \leftarrow W - \frac{W}{(WH)^{\beta - 1} H^\top} \circ \nabla_W D_{\beta}(V, WH)
\end{equation}
This leads to the following update rule:
\begin{equation}
    W \leftarrow W \circ \frac{(V \circ (WH)^{\beta - 2}) H^\top}{(WH)^{\beta - 1} H^\top}
\end{equation}
Ultimately, the updated are given by:
\begin{equation}
    \boxed{W^{(k+1)} = W^{(k)} \circ \frac{(V \circ (W^{(k)}H^{(k)})^{\beta - 2}) H^{(k) \top}}{(W^{(k)}H^{(k)})^{\beta - 1} H^{(k) \top}}}
\end{equation}
and
\begin{equation}
    \boxed{H^{(k+1)} = H^{(k)} \circ \frac{W^{(k) \top} (V \circ (W^{(k)} H^{(k)})^{\beta - 2})}{W^{(k) \top} (W^{(k)} H^{(k)})^{\beta - 1}}}
\end{equation}
\par
The complete MU algorithm is summarized in the following algorithm:
\begin{algorithm}[H]\label{MU for beta}
    \caption{Multiplicative Updates for NMF with the beta divergence}
    \begin{algorithmic}[1]
        \Require \( V \in \mathbb{R}_+^{m \times n} \), rank \( r \), initial factors \( W^{(0)} \in \mathbb{R}_+^{m \times r} \), \( H^{(0)} \in \mathbb{R}_+^{r \times n} \), parameter \( \beta \)
        % \output \( W \in \mathbb{R}_+^{m \times r} \), \( H \in \R_+^{r \times n} \) such that \( V \approx WH \)
        \While{stopping criterion not met}
            \State \( H^{(k+1)} \leftarrow H^{(k)} \circ \frac{W^{(k) \top} (V \circ (W^{(k)} H^{(k)})^{\beta - 2})}{W^{(k) \top} (W^{(k)} H^{(k)})^{\beta - 1}} \)
            \State \( W^{(k+1)} \leftarrow W^{(k)} \circ \frac{(V \circ (W^{(k)}H^{(k+1)})^{\beta - 2}) H^{(k+1) \top}}{(W^{(k)}H^{(k+1)})^{\beta - 1} H^{(k+1) \top}} \)
        \EndWhile
        \State \Return \( W, H \)
    \end{algorithmic}
\end{algorithm}
\subsubsection{Convergence of MU with the $\beta$-divergence}
For the cases where $\beta = 1$ theorem \ref{MU convergence KL} apply, and for $\beta = 2$ theorem \ref{MU convergence Fro} apply.
For others values of $\beta$ we need to adjust MU updated \ref{MU for beta} by a scaling factor $\gamma(\beta)$ \cite{févotte2011algorithmsnonnegativematrixfactorization} defined as:
\begin{equation}
    \gamma(\beta) =
    \begin{cases}
        1/(2 - \beta) & \text{if } \beta < 1 \\
        1 & \text{if } 1 \leq \beta \leq 2 \\
        1/(\beta - 1) & \text{if } \beta > 2
    \end{cases}
\end{equation}
And constraint the iterates to be greater than a small positive constant $\varepsilon > 0$ \cite{takahashi2014global}.
The reason for this is that [...]
\par
Now we are ready to state the more general convergence theorem for MU with the beta divergence:
\begin{theorem}[\cite{gillis2020nonnegative} theorem 8.9]
    Let $\varepsilon > 0$ and consider the adjusted beta divergence Multiplicative updates defined as:
    \begin{align}
        &H^{(k + 1)} \leftarrow \max\Bigg(\varepsilon, 
        H^{(k)} \circ \Big(\frac{W^{(k) \top} (V \circ (W^{(k)} H^{(k)})^{\beta - 2})}{W^{(k) \top} (W^{(k)} H^{(k)})^{\beta - 1}}\Big)^{\gamma(\beta)}\Bigg)
        \\
        &W^{(k + 1)} \leftarrow \max\Bigg(\varepsilon, 
        W^{(k)} \circ \Big(\frac{(V \circ (W^{(k)}H^{(k+1)})^{\beta - 2}) H^{(k+1) \top}}{(W^{(k)}H^{(k+1)})^{\beta - 1} H^{(k+1) \top}}\Big)^{\gamma(\beta)}\Bigg)
    \end{align}
then: 
\begin{enumerate}
    \item The value of the objective function $D(V, WH) = D_{\beta}(V \| WH)$ is non-increasing when $W^{(k)}, H^{(k)}$ are produced by the adjusted beta divergence Multiplicative updates, given that $W \geq \varepsilon$ and $H \geq \varepsilon$.
    \item For any initialization $W^{(0)} \geq \varepsilon$ and $H^{(0)} \geq \varepsilon$, every limit point of the sequence $\{(W^{(k)}, H^{(k)})\}_{k \in \mathbb{N}}$ generated by the adjusted beta divergence Multiplicative updates is a stationary point of the optimization problem $\min_{W,H \geq \varepsilon} D_{\beta}(V \| WH)$.
\end{enumerate}
\end{theorem}
\begin{proof}
    The proof of this theorem can be found in \cite{gillis2020nonnegative} section 8.2.4.
\end{proof}
\subsubsection{Computational cost}

\chapter{Experiments}
In this chapter numerical experiments of NMF algorithms are presented and discussed.

\appendix
\listoffigures
\listofalgorithms
\lstdefinestyle{terminal}{
  backgroundcolor=\color{black},
  basicstyle=\ttfamily\color{white}\small,
  keywordstyle=\color{white},
  commentstyle=\color{white},
  stringstyle=\color{white},
  showstringspaces=false,
  numbers=none,
  frame=single,
  rulecolor=\color{black},
  xleftmargin=0pt,
  xrightmargin=0pt,
  breaklines=true,
  columns=fullflexible
}



\chapter{Code Documentation}\label{code documentation appendix}
% The documentation of the code used in this thesis is provided in this appendix. It includes explanations of the main classes and functions implemented, as well as instructions on how to use the code for reproducing the experiments discussed in the main chapters.   
In this chapter the documentation of the code associated with this thesis is provided, in section \ref{Repo structure section} an overview of the repository content is given, and in section \ref{nmf package section} the main package of the repo is described in details.
Thi chapter concludes showing some examples on how the code can be used.

\par
The code is hosted on GitHub at the following link:
\begin{center}
    \fbox{\url{https://github.com/samueleserri/NMF.git}}
\end{center}
\section{Getting Started}
To get started, clone the repository and navigate to the project directory:
\begin{lstlisting}
git clone https://github.com/samueleserri/NMF.git 
cd NMF
\end{lstlisting}    
or download the ZIP file and extract it.
Then, navigate to the project directory:
\begin{lstlisting}
cd NMF
\end{lstlisting}
\subsection{Install in a virtual environment}
It is recommended to install this package in a virtual environment to avoid conflicts with other packages. To do this, create a virtual environment using \texttt{venv} \footnote{To read more about python virtual environments visit \url{https://docs.python.org/3/library/venv.html}}; open your terminal and (inside the \texttt{NMF} directory) run:
\begin{lstlisting}
python -m venv venv
\end{lstlisting}
Then activate your virtual environment:
\begin{lstlisting}
source venv/bin/activate  # On macOS/Linux
\end{lstlisting}
Finally, inside the virtual environment, install the package in editable mode using pip:
\begin{lstlisting}
pip install -e .
\end{lstlisting}
This will install all required dependencies as well; this command uses the toml file.
To verify the installation, you can run:
\begin{lstlisting}
pip list
\end{lstlisting}
to see the installed packages. You should a package named \texttt{NMF} and its dependencies.

\section{Repository Structure}\label{Repo structure section}

%* begin repo structure
\fbox{%
  \begin{minipage}{\linewidth\fboxsep\fboxrule}
\textbf{NMF/}
    \begin{itemize}
      \item \texttt{src/}
        \begin{itemize}
          \item \texttt{nmf/}
            \begin{itemize}
              \item {\_\_init\_\_.py}
              \item \texttt{NMF.py} \hfill -- base NMF class
              \item \texttt{RegularizedNMF.py} \hfill -- NMF subclass with regularization
              \item \texttt{SparseNMF.py}  \hfill -- NMF subclass for sparse NMF models
              \item \texttt{NonNegMatrix.py} \hfill -- non-negative \texttt{numpy.ndarray}
            \end{itemize}
          \item \texttt{utils/}
            \begin{itemize}
              \item {\_\_init\_\_.py}
              \item \texttt{beta\_divergence.py} \hfill -- compute beta-divergence
              \item \texttt{display.py} \hfill -- helper to display images
              \item \texttt{measure\_sparsity.py} \hfill -- measure matrix sparsity
            \end{itemize}
          \item \texttt{examples/}
            \begin{itemize}
              \item \texttt{minimal\_examples/}
                \begin{itemize}
                  \item \texttt{minimal\_example.py} % \hfill -- basic usage of \texttt{NMF}
                  \item \texttt{example\_regularizer.py} % \hfill -- use \texttt{RegularizedNMF} with a custom regulariser
                \end{itemize}    
                  \item \texttt{example\_feature\_extraction.py} % \hfill -- feature extraction from the CBCL dataset
                  \item \texttt{example\_feature\_extraction\_olivetti\_dataset.py} % \hfill -- feature extraction from the Olivetti dataset
                  \item \texttt{example\_topics\_extraction.py} % \hfill -- topic extraction from the 20‑newsgroups dataset
                  \item \texttt{example\_top\_30.py} % \hfill -- topic modelling with the top‑30 news dataset
            \end{itemize}
        \end{itemize}
      \item \texttt{data/}
        \begin{itemize}
          \item \texttt{CBCL.csv} \hfill -- face dataset used by \cite{Lee1999}
          \item \texttt{Swimmer.csv} \hfill -- 220 $\times$ 256-pixel swimmer images
          \item \texttt{tdt2\_top30.mat} \hfill -- top-30 news dataset for topic modelling
        \end{itemize}
      \item \texttt{tests/}
        \begin{itemize}
          \item \texttt{beta\_divergence\_plot.py} \hfill -- script to plot the beta-divergence
        \end{itemize}
      \item \texttt{README.md}
    \end{itemize}
  \end{minipage}%
}
%* END REPO STRUCTURE

\subsection{The nmf package} \label{nmf package section}
The \texttt{nmf} package contains the main modules with the implementation of the NMF algorithms.
\subsubsection{NMF.py}
The \texttt{NMF.py} module contains the base class for Non-negative Matrix Factorization. This class provides methods for fitting the model to data and plotting the results.
\par
Attributes:
\begin{itemize}
    \item \texttt{V}: input non-negative data matrix.
    \item \texttt{rank}: desired rank for the factorization.
    \item \texttt{max\_iter}: maximum number of iterations for the fitting process. (default: 1000).
    \item \texttt{tol}: tolerance used in the stopping criterion. (default: 1e-4).
    \item \texttt{T}: lag used in the stopping criterion. (default: 10).
    \item \texttt{column\_stochastic}: if True, normalizes the columns of the input matrix to sum to 1. (default: False).
    \item \texttt{init}: Initialization method for factor matrices W and H. Supported values:
        \begin{itemize}
        \item "random": random initialization with uniform distribution in $[0,1)$.
        \item "nndsvd": Non-negative Double Singular Value Decomposition initialization.
        \item "custom": User will set W and H manually.
    \end{itemize} (default random)
    \item \texttt{W0}: Initial matrix for W if init = "custom". (default: None)
    \item \texttt{H0}: Initial matrix for H if init = "custom". (default: None)
\end{itemize}
Public Methods:
\begin{itemize}
    \item \texttt{fit(self, solver: "str", beta: Optional[float] = None)}: fits the NMF model to the input data using the specified solver.
    \item \texttt{plot\_errors()}: plots the normalized reconstruction error on a logarithmic y-scale against iterations.
    \item \texttt{reconstruct()}: reconstructs the input matrix from the factorized matrices.
    \item \texttt{get\_factors()}: getter for the factors matrices W and H.
    \item \texttt{get\_final\_error()}: returns the final normalized reconstruction error.
\end{itemize}
The available solvers are:
\begin{itemize}
    \item \texttt{"MU"}: Multiplicative Updates with Frobenius norm.
    \item \texttt{"HALS"}: Hierarchical Alternating Least Squares.
    \item \texttt{"ALS"}: Alternating Least Squares.
    \item \texttt{"PGD"}: Projected Gradient Descent.
    \item \texttt{"beta\_MU"}: Beta-divergence Multiplicative Updates (requires beta parameter).
\end{itemize}
The solvers are implemented as private methods within the class (double underscore prefix).
\subsubsection{RegularizedNMF}
The \texttt{RegularizedNMF.py} module contains a subclass of the NMF class that allows for regularization in the cost function. This class extends the base NMF class and adds support for $\ell_1$ and $\ell_2$ regularization on the factor matrices $W$ and $H$.
Moreover, it allows for the definition of custom regularizers by the user, which can be added to the cost function and optimized during the fitting process.
\par
Attributes:
\begin{itemize}
    \item \texttt{alpha\_W}: regularization parameter for matrix W.
    \item \texttt{alpha\_H}: regularization parameter for matrix H.
    \item \texttt{regularizer}: a string that specifies the type of regularization to apply. Supported values:
    \begin{itemize}   
        \item "ell\_1": applies $\ell_1$ regularization to both W and H.
        \item "ell\_2": applies $\ell_2$ regularization to both W and H.
        \item "custom": allows the user to define a custom regularizer function.
    \end{itemize}
    \item \texttt{custom\_regularizer}: a user-defined function that takes W and H as input and returns a scalar value representing the regularization term to be added to the cost function. This is used only if regularizer is set to "custom".
\end{itemize}
\subsubsection{SparseNMF}
The \texttt{SparseNMF.py} module contains a subclass of the NMF class that handles sparse matrices with the \texttt{scipy.sparse\_array}\footnote{This is the latest release \url{https://docs.scipy.org/doc/scipy/reference/sparse.html} fully compatible with NumPy arrays, 
the older release was did not have full compatibility, for more information visit \url{https://docs.scipy.org/doc/scipy/reference/sparse.migration_to_sparray.html migration-to-sparray}} interface.
\section{Examples Usage}
\subsection{Basic Usage Example}
First import the NMF class from the nmf package:
\begin{lstlisting}
from nmf import NMF
import numpy as np
\end{lstlisting}
Then create a non-negative data matrix, in this example we generate a random matrix sampled from a uniform distribution:
\begin{lstlisting}
V = np.random.rand(100, 20) @ np.random.rand(20, 100)
\end{lstlisting}
Then instantiate the NMF class with the data matrix and desired rank, in this case we set rank=20 because we constructed V as the product of two matrices of rank at most 20:
\begin{lstlisting}
model = NMF(V, rank=20, max_iter=10000, tol=1e-4, T=10)
\end{lstlisting}
To fit the model we simply call the fit method with the desired solver, in this case we use the Multiplicative Updates (MU) algorithm and the Kullback-Leibler divergence (beta=1):
\begin{lstlisting}
model.fit(solver="beta_MU", beta=1)
print(f"Final error: {model.get_final_error()}")
\end{lstlisting}
You should see the output:
\begin{lstlisting}[style=terminal]
Fitting with beta_MU algorithm
value of beta: 1
Fit completed in 3.8579 s, iterations: 10000, avg time/iter: 3.8579e-04 s
Max iter reached: you may try to increase the value
Final error: 1.33448729119308e-05
\end{lstlisting}
Suppose now that you want to add a regularization term to the Frobenius norm, for instance the $\ell_1$ norm of the factors, then you objective become:
\begin{equation*}
  D(V, WH) = \|V - WH\|_F^2 + \alpha_W \|W\|_1 + \alpha_H \|H\|_1
\end{equation*}
Fitting this model is straightforward: we instantiate a new model from \texttt{RegularizedNMF} specifying the values for $\alpha_W$ and $\alpha_H$, with the desired regularizer, in this case we pass the string \texttt{"ell\_1"} for saying that we want to use the $\ell_1$ norm of the factors as regularization term.
\begin{lstlisting}
reg_model = RegularizedNMF(V, 20, max_iter=1000, alpha_W=0.01, alpha_H=0.01, regularizer="ell_1")  
\end{lstlisting}
Similarly to the previous example, it is sufficient to just call the \texttt{fit} method:
\begin{lstlisting}
reg_model.fit(solver="beta_MU", beta = 2)
\end{lstlisting}
the factors can be retrieved by calling:
\begin{lstlisting}
factors = reg_model.get_factors()
\end{lstlisting}
\par
this return a dictionary where to the key ``W'' corresponds the matrix $W$ and to the key ``H'' corresponds the matrix $H$.
\subsection{Two More Sophisticated Examples} 
\subsubsection{Learning part of faces from the CBCL dataset}
We talk again about the example presented in \ref{Example feature extraction NMF}; recall that this example was originally presented in the seminal paper on NMF by Lee and Seung \cite{Lee1999}. 
Here we reproduce the results using our implementation of NMF with the MU algorithm and the Kullback-Leibler divergence.
\par
First, we need to import the necessary modules: \texttt{pandas}\footnote{\url{https://pandas.pydata.org/docs/}} here is used to read the dataset that is provided as a CSV file in the \texttt{data} folder, the resulting matrix is then converted to a \texttt{np.ndarray};
the function \texttt{display} from the \texttt{utils} package is an helper function and it is used only to visualize the images.
\begin{lstlisting}
import numpy as np
import pandas as pd
from nmf import NonNegMatrix, NMF
from utils import display
\end{lstlisting}
First, we need to load the dataset; to do so we define the following function:
\begin{lstlisting}
def load_dataset(path: str = "data/CBCL.csv") -> NonNegMatrix:
    """
    Load the CBCL face dataset from csv file and return it as NonNegMatrix
    return: NonNegMatrix of shape (361, 2429)
    361 = 19 x 19 pixels
    2429 = number of images
    -------------------
    dataset path: data/CBCL.csv
    -------------------
    """
    return  NonNegMatrix(pd.read_csv(path, header=None).to_numpy())  
\end{lstlisting}
Next, we define a function that: \begin{enumerate}
    \item loads the dataset using the function defined above;
    \item optionally displays the original images;
    \item instantiates and fits the NMF model using the specified solver and rank.
\end{enumerate}
\begin{lstlisting}
def fit_model(rank:int, show: bool = False, solver: str = "beta_MU", beta: float = 1) -> NMF:
    """
    param rank: factorization rank
    param show: shows the original images before fitting the model if true
    param solver: NMF solver to use (in the original paper MU is used)
    return: fitted NMF model
    """ 
    V = load_dataset()    
    if show: # original images displayed if True
        display(V[:, :rank], perrow=7, height=19, width=19)
    # instantiate and fit model
    model = NMF(V, rank)
    model.fit(solver, beta)
    return model    
\end{lstlisting}
Finally, by calling the function just defined, we can fit the model. The rank used in the original paper \cite{Lee1999} is 49, so we set it accordingly:
\begin{lstlisting}
reconstruction_rank = 49
fitted_model = fit_model(reconstruction_rank, solver="beta_MU")
\end{lstlisting}
The $49$ columns of $W$ can be reshaped into $19 \times 19$ images and displayed, to do so we can use the \texttt{display} function again with the same parameters, but this time on the matrix W obtained from the fitted model:
\begin{lstlisting}
display(fitted_model.W[:,:reconstruction_rank], perrow=7, height=19, width=19)
\end{lstlisting}
The output shows the learned parts of faces as in Figure \ref{fig:CBCL columns of W}
\subsubsection{Topic modeling }
TODO

% Bibliography placeholder:
\clearpage
\pagenumbering{roman}
\printbibliography

\end{document}


